\documentclass[11pt]{article}
\usepackage[utf8]{inputenc}
\usepackage{amsmath, amssymb, amsthm}
\usepackage{graphicx}
\usepackage{hyperref}
\usepackage{algorithm}
\usepackage[noend]{algpseudocode}
\usepackage{geometry}
\geometry{margin=1in}
\usepackage{enumitem}
\usepackage{booktabs}
\usepackage{url}

\title{Title of the Paper}
\author{Motoki Nakajima \\ St. Paul's School \\ \texttt{motoki.nakajima@sps.edu}}
\date{\today}

\begin{document}

\maketitle

\section{Proposal}

\subsection{Topic Area}

I select 7. A well-defined algorithmic modification.

\subsection{Formal Research Question}

In a circular harkness table discussion, how effective is the beam search algorithm, compared to a simple greedy algorithm, in determining the table arrangement that maximizes cross-talks between students?

\subsection{Formal Problem Definition}

The problem models a circular harkness table discussion geometrically, where the cross-talk between student $i$ and student $j$ is modeled as $|v_iv_j\cos(\theta_{ij})|$, where $\theta_{ij}$ is the angle between the vector $\overrightarrow{v_i v_j}$ and $\overrightarrow{O v_i}$, and $O$ is the center of the circular table. Because we take the absolute value of the cosine, having talkative people across the table helps active participation. There are $n$ students in the discussion, where each student is given an index of normalized talkativeness $v_i$ ($0 \leq v_i \leq 1$). The goal is to find the arrangement of students around the table that maximizes the total cross-talk, which can be expressed as $\sum_{i\neq j} |v_iv_j\cos(\theta_{ij})|$.

\subsection{Hypothesis}

The beam search algorithm will yield a significantly higher total cross-talk compared to the greedy algorithm, because the greedy algorithm tend to get stuck in an inefficient local maximum, while the beam search explores wider and tend to find a better local maximum. Although the beam search cost more time, considering the small number of students in a typical harkness table discussion, the importance of maximizing the cross-talk performance outweighs the cost of time.

\subsection{Theoretical Plan}

The theoretical analysis will consist of two parts: loop invariant proof that the beam search algorithm will aways find a solution that is at least as good as the greedy algorithm, and a tight bound on the time complexity of the beam search algorithm and the greedy algorithm. The loop invariant proof will be based on the trait of the beam search algorithm that it always explores a wider range of arrangements than the greedy algorithm, while also searching through the same arrangements as the greedy algorithm. The time complexity analysis will be based on the number of arrangements explored by each algorithm, which is determined by the number of students and the beam width. The greedy algorithm has a time complexity of $O(n^2)$, while the beam search algorithm has a time complexity of $O(B \cdot n^2)$, where $B$ is the beam width. 

\subsection{Experimental Plan}

The input will be a vector of $n$ normalized talkativeness values $v_i$, which all of them will be generated randomly through \texttt{std::random\_device} in C++. The single array of input will automatically determine the size of the harkness table and the students' talkativeness, and thus completely determine all the parameters of the problem. The test will be conducted for various sizes of $n$ from 5 to 100, with increments of 5, and for various beam widths $B$ from 1 to 10, with increments of 1. 20 trials will be conducted for each combination of $n$ and $B$, and the average total cross-talk will be recorded for both the greedy algorithm and the beam search algorithm. The total cross-talk index will then be plotted by \texttt{matplotlib} in Python in three different ways: a 3d plot of total cross-talk against $n$ and $B$, a 2d plot of total cross-talk against $n$ for a fixed $B$ where the difference between the two algorithms are at largest, and a 2d plot of total cross-talk against $n$ for a fixed $B$ where the difference is smallest. The results will be analyzed to determine the effectiveness of the beam search algorithm compared to the greedy algorithm in maximizing the total cross-talk in a circular harkness table discussion.

\bibliographystyle{plain}
\bibliography{references}

\end{document}
